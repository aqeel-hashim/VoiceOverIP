\documentclass[10pt,a4paper]{article}
\usepackage{amsmath}
\usepackage{amsfonts}
\usepackage{amssymb}
\usepackage{graphicx}
\usepackage{amstext}
\usepackage{url}
\usepackage[table]{xcolor}
\usepackage[font=scriptsize, center]{caption}



\title{VoIP Chat Program}
\author{R. van Tonder, 15676633 \\ A. Esterhuizen, 15367940}
\date{\today}

\begin{document}
\maketitle
\newpage
\tableofcontents
\newpage

\section{Description}
\paragraph{} This project comprised of implementing a Voice over IP chat program. It enables
users to engage in a VoIP conversation over a local network, as well as chat. The architecture
is that of a client-server model. The server concerns itself primarily with introducing clients to one another.

\paragraph{} Consider that host A wishes to call host B, but that it has no knowledge of host B's address and port.
The server will provide this information to host A, such that host A and B can initiate a direct connection
and proceed to a VoIP conversation. Furthermore, chat messages from clients are relayed via the server.

\paragraph{} The client is the front-end interface that initiates call operations with other hosts. Upon connecting to
the server, it provides it's host-name. The server can then provide this information to other hosts wishing
to engage in VoIP. The client provides a GUI which updates messages received, as well as a user-list maintained
by the server.

\section{Features}
%included
\subsection{Included}
\subsubsection{Server}
The server has the following capabilities, in accordance with the
project specification.

\begin{enumerate}
 \item The server accepts connection requests from clients, and updates the user-list with the host-name of connected clients.
 \item When clients wish to initiate a voice conversation, the server responds with the appropriate host-name of the recipient. 
In this manner clients receive permission to call other hosts.
 \item Informs clients if a call or conference is already active between hosts.
 \item The server relays all messages to all clients, if they are not recognized as a command such as \verb|\call|. See \ref{det} for more.
 \item Multiple users can connect simultaneously.
 \item Whispering and global chat message relaying.
 \item Call and conference channels. This is done by keeping a list of current active calls and conferences. See \ref{det} for more.
 \item A GUI which displays 
 \begin{itemize}
  \item All messages sent through the server
  \item Client connections to the server
  \item Client disconnections from the server
  \item The current user-list
  \item Responses to commands received by the server
 \end{itemize}

\end{enumerate}

\subsubsection{Client}
The client has the following capabilities, in accordance with the project specification.

\begin{enumerate}
 \item Clients may connect and disconnect without incident.
 \item Commands that are processed by the server for various functionalities, such as \verb|\call|, \verb|\callc|, and \verb|\msg|. See \ref{det} for more.
 \item Initiating voice transmission when in a call with another host.
 \item Playing Voice output and receiving Mic input.
 \item A GUI which displays
 \begin{itemize}
  \item The current user-list
  \item All global messages, and whispers when applicable
  \item Whether a call has been iniatiated with the client
  \item Whether a call cannot be established, if a host is already in a conference or call.
 \end{itemize}

\end{enumerate}

%not included - nothing to say for now
%extra features - nothing to say for now

%See \ref{comp}

\section{Design}
\label{det}
\subsection{Server Design}
The server implementation was written in Python. It parses all incoming messages for the following commands,
and performs the appropriate action.

\begin{itemize}
 \item \verb|\call <host-name>| Initiate a voice conversation with a host. The server responds with the host IP, or a message that a call is already in progress with this host.
 \item \verb|\callc <host-name>| Initiate a voice conversation with multiple hosts who are already in a call. The conversation will be carried out with hosts which belong to \verb|host-name|'s conference.
 \item \verb|\msg <host-name> <message>| Whisper a message to a host-name which exists.
 \item \verb|\dc| Disconnect from a call or conference if engaged in one.
\end{itemize}

All messages received by the server that do not conform to these commands are broadcast to all clients as a message.
\paragraph{}
Furthermore, the server maintains a list of threads assocaited with active connections,
and a 2-dimensional list of calls and conferences currently in progress. For example, consider the 2-dimensional list:
\paragraph{}
\verb|[[146.232.50.1, 146.232.50.20, 146.232.50.41], [146.232.50.5, 146.232.50.81]]|
\paragraph{}
This is representative of the fact that hosts \verb|146.232.50.1, 146.232.50.20,| and \verb|146.232.50.41| are in a conference call,
while hosts \verb|146.232.50.5| and \verb|146.232.50.81| are in a seperate call. The call/conference list can grow indefinitely,
as well as the hosts within a conference. All this information is kept server side, and although clients are informed of
connecting hosts, they maintain no data of the hosts in the call/conference.
\paragraph{}
The server sends an updated user-list to all clients when a change in the user-list takes place. This is done by serializing the
host-names associated with the  active connections list of thread objects.
\paragraph{}
The server GUI was written in PyQt.

\subsection{Client Design}

%threads? invoke lateR? etc
%in Java
%how do sound?
%updating of userlist? It makes use of 'picked' objects, just say that. remember that client only receives from server, no need to
%serialize data as it doesnt send it.

\section{Sound Quality} %bs everything

\section{Complications} %as jy lus het
\label{comp}

Linux sound libraries shit.

\section{Conclusion}
%goals achieved, overall wrap-up. voip is nie 'n lekker ou nie.



\end{document}
